% Copyright (C) 2012 Raniere Silva
%
% This work is licensed under the Creative Commons
% Attribution-ShareAlike 3.0 Unported License. To view a copy of this
% license, visit <http://creativecommons.org/licenses/by-sa/3.0/>.
%
% This work is distributed in the hope that it will be useful, but
% WITHOUT ANY WARRANTY; without even the implied warranty of
% MERCHANTABILITY or FITNESS FOR A PARTICULAR PURPOSE.

\documentclass[11pt]{beamer}
% Utilizar apenas para a classe beamer
\usetheme{CambridgeUS}
\let\Tiny=\tiny % Redefine at least \Tiny for avoid warning

% Tipo de arquivo.
\usepackage[utf8]{inputenc}
% \usepackage[latin1]{inputenc}
\usepackage[T1]{fontenc}

% Configura\c{c}\~{o}es regionais
% \usepackage[top=3cm,left=2cm,right=2cm,bottom=3cm]{geometry}
\usepackage[brazil]{babel}
\usepackage{indentfirst}
\uselanguage{brazil}
\languagepath{brazil}
\deftranslation[to=brazil]{Example}{Exemplo}

% Textos
\newcommand{\flang}[1]{\textit{#1}}

% Links
\usepackage{url}
\usepackage{hyperref}
\hypersetup{
%colorlinks = true,
}
\usepackage{breakurl}

% Pacotes matem\'{a}ticos
\usepackage{amsmath}
\usepackage{amsfonts}
\usepackage{amssymb}
\usepackage{amsthm}
\allowdisplaybreaks[4]
\newtheorem{defi}{Definição}
\newtheorem{prop}{Proposição}

% Pacotes para tabelas
\usepackage{multicol}
\usepackage{multirow}
\usepackage{array}

% Pacotes gr\'{a}ficos
\usepackage{graphicx}
\usepackage{subfigure}
\usepackage{wrapfig}
\usepackage{tikz}
\usetikzlibrary{positioning}
\usetikzlibrary{fit}
\usetikzlibrary{patterns}%verificar se necessário

% Pacotes para algoritmos
\usepackage{algorithmic}
\algsetup{linenosize=\small}
\renewcommand{\algorithmicrequire}{\textbf{Entrada:}}
\renewcommand{\algorithmicensure}{\textbf{Saída:}}
\renewcommand{\algorithmicend}{\textbf{fim}}
\renewcommand{\algorithmicif}{\textbf{se}}
\renewcommand{\algorithmicthen}{\textbf{ent\~{a}o}}
\renewcommand{\algorithmicelse}{\textbf{caso contr\'{a}rio}}
\renewcommand{\algorithmicendif}{\algorithmicend}
\renewcommand{\algorithmicfor}{\textbf{para}}
\renewcommand{\algorithmicforall}{\textbf{para todo}}
\renewcommand{\algorithmicdo}{\textbf{fa\c{c}a}}
\renewcommand{\algorithmicendfor}{\algorithmicend}
\renewcommand{\algorithmicwhile}{\textbf{enquanto}}
\renewcommand{\algorithmicendwhile}{\algorithmicend}
\renewcommand{\algorithmicrepeat}{\textbf{repita}}
\renewcommand{\algorithmicuntil}{\textbf{at\'{e}}}
\renewcommand{\algorithmicreturn}{\textbf{retorne}}
\renewcommand{\algorithmiccomment}[1]{\hspace{2em}/* #1 */}

\usepackage{algorithm}
\floatname{algorithm}{Algoritmo}

% Pacotes para c\'{o}digos
\usepackage{textcomp}
\usepackage{listings}
\renewcommand{\lstlistingname}{C\'{o}digo}
\lstset{
% language=Octave,
basicstyle=\ttfamily\scriptsize,
columns=flexible,
% numbers=left,
% numberstyle=\footnotesize,
% stepnumber=5,
% numbersep=5pt,
% backgroundcolor=\color{white},
% showspaces=false,
% showstringspaces=false,
% showtabs=false,
% frame=single,
tabsize=4,
captionpos=t,
breaklines=true,
breakatwhitespace=false,
% caption={\texttt{\lstname}},
% escapeinside={\%*}{*)},
% morekeywords={#},
upquote=true,
}
\newcommand{\lcode}[1]{\lstinline!#1!}
% Configura\{c}c\~{o}es para o python
\lstdefinestyle{python}{
language=python,
% escapeinside={\%}{\^{M},
}
\lstnewenvironment{cpython}{\lstset{style=python,}}{}
\newcommand{\fpython}[1]{\lstinputlisting[style=python,]{#1}}

% Index
\usepackage{makeidx}
\makeindex

\begin{document}
\title{MS480 - Problema do Aterro}
\author[Silva, Cezarino, Marques e Lima]{Raniere Gaia Costa da Silva\footnote{ra092767,
\url{r.gaia.cs@gmail.com}} \and Fernando de Oliveira Cezarino\footnote{ra085855,
\url{feolce@gmail.com}} \and Ana~Paula Diniz Marques\footnote{ra076433,
\url{anapdinizm @gmail.com}} \and Ana Flávia da Cunha Lima\footnote{ra093370,
\url{anaflavia.c.lima @gmail.com}}}

\begin{frame}
    \maketitle
\end{frame}

\begin{frame}
    \begin{block}{}
        Os arquivos desta apresentação encontram-se disponíveis \\
        em \url{https://github.com/r-gaia-cs/ms480-2012s2-aterro}.
    \end{block}

    \begin{block}{Licença}
        Salvo indicado o contrário, esta apresentação está licenciada sob a licença
        Creative Commons Atribuição 3.0 Não Adaptada. Para ver uma cópia desta
        licença, visite http://creativecommons.org/licenses/by/3.0/.
        \begin{center}
            \includegraphics{../figuras/cc-by.png}
        \end{center}
    \end{block}
\end{frame}

\begin{frame}
    \tableofcontents
\end{frame}

\section{Problema do Aterro}
% \begin{frame}
% O problema deste trabalho denominado de ``Aterro com obstáculo'', é uma
% versão modificada do problema abordado por Enrique D. Andjel, Tarcísio L. Lopes
% e José Mario Martinez em ``A Linear Continuous Transportation Problem''
% \cite{Andjel:1989:TP}.
% \end{frame}

\begin{frame}{Região}
    \begin{center}
        \begin{tikzpicture}[scale=.75]
            \draw[pattern=bricks] (0,0) -- (0,3) -- (2,2) -- (4,0) -- (0,0);
            \node[fill=white] at (1,1) {$A$};
            \draw[pattern=checkerboard] (6,6) -- (8,6) -- (8,8) -- (6,8) -- (5,7) --
            (6,6);
            \node[fill=white] at (7,7) {$J$};
            \only<2>{
            \draw[line width=4, green] (1.8,1.8) -- (5.6,7.2);
            \draw[line width=4, red] (3.6,.2) -- (7.8,7.8);
            }

            % Escala
            \node[fill=white, above] at (5,1.2) {$0$};
            \draw[fill=black] (5,1) rectangle (6,1.2) node[fill=white, above]{$e$};
            \draw[fill=white] (6,1) rectangle (7,1.2) node[fill=white, above]{$2e$};
            \draw[fill=black] (7,1) rectangle (8,1.2) node[fill=white, above]{$3e$};

            % Eixos
            \draw[->] (-.2,8) -- (8.2,8);
            \draw[->] (0,8.2) -- (0,-.2);
        \end{tikzpicture}
    \end{center}
\end{frame}

\section{Problema do Aterro com Obstáculo}
\begin{frame}{Região com obstáculo}
    \begin{center}
        \begin{tikzpicture}[scale=.75]
            \draw[pattern=bricks] (0,0) -- (0,3) -- (2,2) -- (4,0) -- (0,0);
            \node[fill=white] at (1,1) {$A$};
            \draw[pattern=checkerboard] (6,6) -- (8,6) -- (8,8) -- (6,8) -- (5,7) --
            (6,6);
            \node[fill=white] at (7,7) {$J$};
            \draw[pattern=fivepointed stars] (4,4) node[fill=white]{$R$}
            circle (1.7) ;
            \only<2>{
            \draw[line width=4, green] (.8,1.8) -- (2,6) -- (5.6,7.2);
            \draw[line width=4, red] (.4,.4) -- (.4,7) -- (7.8,7.8);
            }

            % Escala
            \node[fill=white, above] at (5,1.2) {$0$};
            \draw[fill=black] (5,1) rectangle (6,1.2) node[fill=white, above]{$e$};
            \draw[fill=white] (6,1) rectangle (7,1.2) node[fill=white, above]{$2e$};
            \draw[fill=black] (7,1) rectangle (8,1.2) node[fill=white, above]{$3e$};

            % Eixos
            \draw[->] (-.2,8) -- (8.2,8);
            \draw[->] (0,8.2) -- (0,-.2);
        \end{tikzpicture}
    \end{center}
\end{frame}

\section{A Modelagem}
\begin{frame}{Discretização}
    \begin{center}
        \begin{tikzpicture}[scale=.75]
            \draw[color=gray, step=.5] (0,0) grid (8,8);
            \draw[pattern=bricks] (0,0) -- (0,3) -- (2,2) -- (4,0) -- (0,0);
            \node[fill=white] at (1,1) {$A$};
            \draw[pattern=checkerboard] (6,6) -- (8,6) -- (8,8) -- (6,8) -- (5,7) --
            (6,6);
            \node[fill=white] at (7,7) {$J$};

            \draw[pattern=bricks](0,0) rectangle (2,2) node[fill=white,
            midway]{$A$};
            \draw[pattern=checkerboard] (6,6) rectangle (8,8) node[fill=white,
            midway]{$J$};

            \only<2>{
            \foreach \x in {1,...,11}{
                \node at (\x/2 - .25,7.75) {\small{$\x$}};
            }
            \foreach \x in {17,...,26}{
                \node at (\x/2 - 8.25,7.25) {\small{$\x$}};
            }
            }

            % Escala
            \node[fill=white, above] at (5,1.2) {$0$};
            \draw[fill=black] (5,1) rectangle (6,1.2) node[fill=white, above]{$e$};
            \draw[fill=white] (6,1) rectangle (7,1.2) node[fill=white, above]{$2e$};
            \draw[fill=black] (7,1) rectangle (8,1.2) node[fill=white, above]{$3e$};

            % Eixos
            \draw[->] (-.2,8) -- (8.2,8);
            \draw[->] (0,8.2) -- (0,-.2);
        \end{tikzpicture}
    \end{center}
\end{frame}

\begin{frame}{Ajuste}
    \begin{center}
        \begin{tikzpicture}[scale=.75]
            \draw[color=gray, step=.5] (0,0) grid (8,8);
            \draw[pattern=bricks] (0,0) -- (0,3) -- ++(1,0) -- ++(0,-.5) -- ++(1,0)
            -- ++(0,-.5) -- ++(.5,0) -- ++(0,-.5) -- ++(.5,0) -- ++(0,-.5) --
            ++(.5,0) -- ++(0,-.5) -- ++(.5,0) -- ++(0,-.5) -- (4,0) -- (0,0);
            \node[fill=white] at (1,1) {$A$};
            \draw[pattern=checkerboard] (6,6) -- (8,6) -- (8,8) -- (6,8) --
            ++(-.5,0) -- ++(0,-.5) -- ++(-.5,0) -- ++ (0,-.5) -- ++(0,-.5) --
            ++(.5,0) -- ++(0,-.5) -- ++(.5,0) -- (6,6);
            \node[fill=white] at (7,7) {$J$};

            \draw[pattern=bricks](0,0) rectangle (2,2) node[fill=white,
            midway]{$A$};
            \draw[pattern=checkerboard] (6,6) rectangle (8,8) node[fill=white,
            midway]{$J$};

            % Escala
            \node[fill=white, above] at (5,1.2) {$0$};
            \draw[fill=black] (5,1) rectangle (6,1.2) node[fill=white, above]{$e$};
            \draw[fill=white] (6,1) rectangle (7,1.2) node[fill=white, above]{$2e$};
            \draw[fill=black] (7,1) rectangle (8,1.2) node[fill=white, above]{$3e$};

            % Eixos
            \draw[->] (-.2,8) -- (8.2,8);
            \draw[->] (0,8.2) -- (0,-.2);
        \end{tikzpicture}
    \end{center}
\end{frame}

\begin{frame}{Modelagem problema com obstáculo circular}
    \begin{align*}
        \text{max } & \sum_{x} \sum_{z} \sum_{y} \xi_{x, z, y}, \\
        \text{s.a. } & \xi_{x, z, y} \geq 0, && \forall x, z, y, \\
        & \xi_{x, z, y} = 0, && \forall (x, z, y) \mid d(x, z) > D \text{ ou } 
        d(y, z) > D, \\
        & \xi_{x, z, y} = 0, && \forall (x, z, y) \mid l(x, z) \leq r \text{ ou }
        l(y, z) \leq r, \\
        & \sum_{z} \sum_{y} \xi_{x, z, y} \leq \phi(x), && \forall x, \\
        & \sum_{x} \sum_{z} \xi_{x, z, y} \leq \psi(y), && \forall y,
    \end{align*}
    onde $\phi(x)$ é o volume de terra disponível em $x$, $\psi(y)$ é o volume de
    terra admitido em $y$; $\xi_{x, z, y}$ é o volume de terra transportado de $x$
    para $y$ passando por $z$.
\end{frame}

\begin{frame}{Problema com obstáculo geral}
    \begin{center}
        \begin{tikzpicture}[scale=.75]
            \draw[color=gray, step=.5] (0,0) grid (8,8);

            \draw[color=gray, step=.5] (0,0) grid (8,8);
            \draw[pattern=bricks] (0,0) -- (0,3) -- ++(1,0) -- ++(0,-.5) -- ++(1,0)
            -- ++(0,-.5) -- ++(.5,0) -- ++(0,-.5) -- ++(.5,0) -- ++(0,-.5) --
            ++(.5,0) -- ++(0,-.5) -- ++(.5,0) -- ++(0,-.5) -- (4,0) -- (0,0);
            \node[fill=white] at (1,1) {$A$};
            \draw[pattern=checkerboard] (6,6) -- (8,6) -- (8,8) -- (6,8) --
            ++(-.5,0) -- ++(0,-.5) -- ++(-.5,0) -- ++ (0,-.5) -- ++(0,-.5) --
            ++(.5,0) -- ++(0,-.5) -- ++(.5,0) -- (6,6);
            \node[fill=white] at (7,7) {$J$};
            \draw[pattern=fivepointed stars] (3,2.5) -- ++(0,-.5) -- ++(2,0) --
            ++(0,.5) -- ++ (.5,0) -- ++(0,.5) -- ++(.5,0) -- ++(0,2) -- ++ (-.5,0)
            -- ++(0,.5) -- ++(-.5,0) -- ++(0,.5) -- ++(-2,0) -- ++(0,-.5) --
            ++(-.5,0) -- ++(0,-.5) -- ++(-.5,0) -- ++(0,-2) -- ++(.5,0) -- ++(0,-.5)
            -- (3,2.5);
            \node[fill=white] at (4,4) {$R$};

            % Escala
            \node[fill=white, above] at (5,1.2) {$0$};
            \draw[fill=black] (5,1) rectangle (6,1.2) node[fill=white, above]{$e$};
            \draw[fill=white] (6,1) rectangle (7,1.2) node[fill=white, above]{$2e$};
            \draw[fill=black] (7,1) rectangle (8,1.2) node[fill=white, above]{$3e$};

            % Eixo
            \draw[->] (-.2,8) -- (8.2,8);
            \draw[->] (0,8.2) -- (0,-.2);
        \end{tikzpicture}
    \end{center}
    \begin{align*}
        \xi_{x, z, y} &= 0, && \forall (x, z, y) \mid b(x, z) = 1 \text{ ou }
        b(y, z) = 1,
    \end{align*}
\end{frame}

\begin{frame}{Modelagem Aperfeiçoada}
    Consideremos agora os pontos $\hat{x} \in J$, $\hat{y} \in A$, $\hat{z},
    \tilde{z} \in H$, tal que
    \begin{align*}
        d(\hat{x}, \hat{z}) &\leq D, & d(\hat{y}, \hat{z}) &\leq D, \\
        l(\hat{x}, \hat{z}) &> r, & l(\hat{y}, \hat{z}) &> r, \\
        d(\hat{x}, \tilde{z}) &\leq D, & d(\hat{y}, \tilde{z}) &\leq D, \\
        l(\hat{x}, \tilde{z}) &> r, & l(\hat{y}, \tilde{z}) &> r
    \end{align*}
    e
    \begin{align*}
        d(\hat{x}, \hat{z}) + d(\hat{y}, \hat{z}) &< d(\hat{x}, \tilde{z}) +
        d(\hat{y}, \tilde{z}).
    \end{align*}

    \pause
    Verifica-se que
    $\xi_{\hat{x},\hat{z},\hat{y}}$ e $\xi_{\hat{x},\tilde{z},\hat{y}}$
    possuem o mesmo peso na fun\c{c}\~{a}o objetivo e portanto os seguintes valores
    s\~{a}o equiprov\'{a}veis na solu\c{c}\~{a}o \'{o}tima:
    \begin{align*}
        \xi_{\hat{x},\hat{z},\hat{y}} &= \min(\phi(\hat{x}), \psi(\hat{y})), &
        \xi_{\hat{x},\tilde{z},\hat{y}} &= 0, \\
        \xi_{\hat{x},\hat{z},\hat{y}} &= 0, &
        \xi_{\hat{x},\tilde{z},\hat{y}} &= \min(\phi(\hat{x}), \psi(\hat{y})), \\
        \xi_{\hat{x},\hat{z},\hat{y}} &= 0.5 \min(\phi(\hat{x}), \psi(\hat{y})), &
        \xi_{\hat{x},\tilde{z},\hat{y}} &= 0.5 \min(\phi(\hat{x}), \psi(\hat{y})).
    \end{align*}
\end{frame}

\begin{frame}{Minimizando a distância}
    Resolver o seguinte problema de Programação Não-Linear:
    \begin{align*}
        \underset{Z}{\text{min }} & \sqrt{(Z_1 - X_1)^2 + (Z_2 - X_2)^2} +
        \sqrt{(Z_1 - Y_1)^2 + (Z_2 - Y_2)^2}, \\
        \text{s.a. } & \Delta(X, Z) < \epsilon, \\
        & \Delta(Y, Z) < \epsilon,
    \end{align*}
    onde
    \begin{align*}
        \Delta(P, Q) = r^2 \left( (Q_1 - P_1)^2 + (Q_2 - P_2)^2 \right) - \left(
        P_1 Q_2 - P_2 Q_1 \right)^2,
    \end{align*}
    e $X, Y, Z$ são o par coordenado correspondente a posição do centro dos
    retângulos $x, y, z$, respectivamente, transladada para que o centro da região
    $R$ seja $(0, 0)$, e $r$ é o raio da região $R$.
\end{frame}

\section{Implementação}
\begin{frame}{Softwares Utilizados}
A parte computacional do projeto encontra-se disponível em
\url{https://github.com/r-gaia-cs/ms480-2012s2-aterro/tree/master/src} e foi
desenvolvida utilizando:
\begin{itemize}
    \item Python\nocite{Python},
    \item SciPy\nocite{SciPy},
    \item GLPK (GNU Linear Programming Kit)\nocite{GLPK},
    \item Python-GLPK.
\end{itemize}
Para a criação dos testes, utilizou-se
\begin{itemize}
    \item Inkscape e
    \item ImageMagick
\end{itemize}
e o formato \texttt{ppm} para armazenar os mapas dos testes.
\end{frame}

\section{Resultados}
\begin{frame}{Solver}
\begin{table}
    \begin{tabular}{|c|c|c|c|c|c|}
        \hline
        & \multicolumn{5}{|c|}{t1.svg} \\ \hline
        Solver & Linhas & Colunas & F. Obj. & T. Proc. & T. Sol. \\ \hline
        Sem obstáculo & 110 & 817 & 38.0 & 3.5974 & 0.0436 \\ \hline
        Obstáculo circular & 110 & 1111 & 45.0 & 3.5443 & 0.0635 \\ \hline
        Obstáculo geral & 110 & 2989 & 49.0 & 3.5674 & 0.1562  \\ \hline
    \end{tabular}
\end{table}
\end{frame}

\begin{frame}{Tipo de distância}
\begin{block}{}
O Teorema~1 em \cite{Andjel:1989:TP}, diz que
\begin{align*}
    F(d^l) \geq F(d^c) \geq F(d^u),
\end{align*}
onde $F(d)$ é o valor da função objetivo ao utilizar a distância do tipo $d$.
\end{block}

\begin{table}
    \centering
    \begin{tabular}{|c|c|c|c|c|c|}
        \hline
        & \multicolumn{5}{|c|}{t1.svg}  \\ \hline
        Tipo Dist. & Linhas & Colunas & F. Obj. & T. Proc. & T. Sol. \\ \hline
        $d^l$ & 110 & 2034 & 49.0 & 3.5 & 0.1101 \\ \hline
        $d^c$ & 110 & 1578 & 49.0 & 3.4 & 0.0870 \\ \hline
        $d^u$ & 110 & 1111 & 45.0 & 3.5 & 0.0635 \\ \hline
    \end{tabular}        
\end{table}
\end{frame}

\begin{frame}{Comprimento Máximo dos Canos}
\begin{table}
    \centering
    \begin{tabular}{|c|c|c|c|c|c|}
        \hline
        & \multicolumn{5}{|c|}{t1.svg} \\ \hline
        Max. Dist. & Linhas & Colunas & F. Obj. & T. Proc. & T. Sol.  \\ \hline
        800.0 & 110 & 1111 & 45.0 & 3.5443 & 0.0635  \\ \hline
        900.0 & 110 & 1776 & 49.0 & 3.4865 & 0.0959  \\ \hline
        1000.0 & 110 & 2404 & 49.0 & 3.6357 & 0.1289  \\ \hline
    \end{tabular}
\end{table}
\end{frame}

\begin{frame}{Discretização}
\begin{table}
    \centering
    \begin{tabular}{|c|c|c|c|c|c|c|c|c|c|c|}
        \hline
        & \multicolumn{5}{|c|}{t1.svg}  \\ \hline
        Redução & Linhas & Colunas & F. Obj. & T. Proc. & T. Sol. \\ \hline
        100 & 34 & 101 & 14.0 & 3.6675 & 0.0059 \\ \hline
        50 & 110 & 1111 & 45.0 & 3.5443 & 0.0635 \\ \hline
    \end{tabular}
\end{table}
\end{frame}

\section*{Obrigado}
\begin{frame}
    \begin{center}
        Obrigado!
    \end{center}
\end{frame}

\section*{Bibliografia}
\begin{frame}{Bibliografia}
\bibliographystyle{alpha}
\bibliography{../referencias}
\end{frame}
\end{document}
