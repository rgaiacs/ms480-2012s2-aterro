% Copyright (c) 2012 Raniere Silva <r.gaia.cs@gmail.com>
% Copyright (c) 2012 Fernando Cezarino <feolce@gmail.com>
% Copyright (c) 2012 Ana Paula Diniz Marques <anapdinizm@gmail.com>
% Copyright (c) 2012 Camile Kunz <camileknz@gmail.com>
% Copyright (c) 2012 Ana Flavia <anaflavia.c.lima@gmail.com>
%
% This file is part of 'MS480 - 2012S2 - Aterro com Obstáculo'.
%
% 'MS480 - 2012S2 - Aterro com Obstáculo' is licensed under the Creative
% Commons Attribution-ShareAlike 3.0 Unported License. To view a copy of
% this license, visit http://creativecommons.org/licenses/by-sa/3.0/.
%
% 'MS480 - 2012S2 - Aterro com Obstáculo' is distributed in the hope
% that it will be useful, but WITHOUT ANY WARRANTY; without even the
% implied warranty of MERCHANTABILITY or FITNESS FOR A PARTICULAR
% PURPOSE.

\section{Formato Netpbm} \label{sse:Netpbm}
Nesta seção descrevemos o formato de arquivo \clang{ppm} utilizado neste projeto
que é um mapa de pixels capaz de representar todas as cores definidas pelo
padrão \clang{RGB} e um dos formatos descritos na biblioteca
Netpbm.\nocite{wiki:Netpbm_format, Henderson:Netpbm}

Cada arquivo inicia com dois bits (em ASCII) que especifica o formato do arquivo
e a codificação (ASCII ou binário), no caso do arquivo \clang{ppm} utilizados
será \clang{P3}.\footnote{Linhas iniciadas com cerquilha, \#, são tratadas
como comentários.}

Após os dois bits iniciais, é especificado a largura e a altura da imagem e o
valor máximo das cores, que deve ser maior que zero e menor que 65536. Por
último, é apresentado um conjunto de triplas que indicam os valores das cores
vermelho, verde e azul do pixel corespondente, sendo que a ordem de relação
entre as triplas e os pixels é da esquerda para a direita e de cima para baixo.

A seguir é apresentado o conteudo de um arquivo \clang{ppm}. A visualização do
mesmo encontra-se na Figura~\ref{fig:minimal.ppm}.
\lstinputlisting[basicstyle=\ttfamily,]{../src/test/minimal.ppm}
\begin{figure}[!htb]
    \begin{center}
        \includegraphics[scale=20]{../src/test/minimal.png}
    \end{center}
    \caption{Visualização do arquivo \clang{ppm} apresentado na
    Seção~\ref{sse:Netpbm} com um aumento de 20 vezes.}
    \label{fig:minimal.ppm}
\end{figure}

