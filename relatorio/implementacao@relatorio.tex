% Copyright (c) 2012 Raniere Silva <r.gaia.cs@gmail.com>
% Copyright (c) 2012 Fernando Cezarino <feolce@gmail.com>
% Copyright (c) 2012 Ana Paula Diniz Marques <anapdinizm@gmail.com>
% Copyright (c) 2012 Camile Kunz <camileknz@gmail.com>
% Copyright (c) 2012 Ana Flavia <anaflavia.c.lima@gmail.com>
%
% This file is part of 'MS480 - 2012S2 - Aterro com Obstáculo'.
%
% 'MS480 - 2012S2 - Aterro com Obstáculo' is licensed under the Creative
% Commons Attribution-ShareAlike 3.0 Unported License. To view a copy of
% this license, visit http://creativecommons.org/licenses/by-sa/3.0/.
%
% 'MS480 - 2012S2 - Aterro com Obstáculo' is distributed in the hope
% that it will be useful, but WITHOUT ANY WARRANTY; without even the
% implied warranty of MERCHANTABILITY or FITNESS FOR A PARTICULAR
% PURPOSE.

\section{Implementa\c{c}\~{a}o computacional}
A parte computacional do projeto foi desenvolvida utilizando a linguagem de
programação Python\nocite{Python}, o pacote Scipy, o GLPK (GNU Linear
Programming Kit)\nocite{GLPK} e o Python-GLPK.
%TODO usarm emph?

As figuras utilizadas para armazenar os mapas foram armazenadas
utilizando o formato \texttt{ppm}.

\subsection{Implementação do problema sem obstáculo}
Ao implementar \eqref{eq:model_without_obs} foi possível resolver em
% TODO Incluir resultados.

O arquivo \texttt{ppm} é lido e ''comprimido'' de acordo com o tamanho da malha
desejada. Depois, construímos o modelo e por último, resolvemos o problema.
%TODO melhorar esse parágrafo

\subsection{Implementação do problema com obstáculo circular}
Para o subproblema de determinar o ponto $z$ que minimiza $\overline{xyz}$,
utilizou-se a função \texttt{fmin-cobyla} do pacote Scipy, que minimiza uma
função de várias variáveis sujeita a restrições utilizando um algoritmo baseado
em aproximações lineares para a função objetivo e as restrições. Maiores
informações sobre a função \texttt{fmin-cobyla} podem ser encontradas no
Apêndice , na documentação do Scipy e em.
%TODO colocar referências

Depois, disso, prosseguimos como no problema dem obstáculo.

\subsection{Implementação do problema com obstáculo geral}
Utilizou-se o algoritmo de Bresenham.

%TODO melhorar essa seção