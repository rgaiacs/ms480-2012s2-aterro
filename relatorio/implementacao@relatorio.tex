% Copyright (c) 2012 Raniere Silva <r.gaia.cs@gmail.com>
% Copyright (c) 2012 Fernando Cezarino <feolce@gmail.com>
% Copyright (c) 2012 Ana Paula Diniz Marques <anapdinizm@gmail.com>
% Copyright (c) 2012 Camile Kunz <camileknz@gmail.com>
% Copyright (c) 2012 Ana Flavia <anaflavia.c.lima@gmail.com>
%
% This file is part of 'MS480 - 2012S2 - Aterro com Obstáculo'.
%
% 'MS480 - 2012S2 - Aterro com Obstáculo' is licensed under the Creative
% Commons Attribution-ShareAlike 3.0 Unported License. To view a copy of
% this license, visit http://creativecommons.org/licenses/by-sa/3.0/.
%
% 'MS480 - 2012S2 - Aterro com Obstáculo' is distributed in the hope
% that it will be useful, but WITHOUT ANY WARRANTY; without even the
% implied warranty of MERCHANTABILITY or FITNESS FOR A PARTICULAR
% PURPOSE.

\section{Implementa\c{c}\~{a}o computacional}
A parte computacional do projeto encontra-se disponível em
\url{https://github.com/r-gaia-cs/ms480-2012s2-aterro/tree/master/src} e foi
desenvolvida utilizando:
\begin{itemize}
    \item Python\nocite{Python},
    \item Scipy\nocite{SciPy},
    \item GLPK (GNU Linear Programming Kit)\nocite{GLPK},
    \item Python-GLPK.
\end{itemize}
O processamento dos testes para a construção dos elementos do problema é feita
em Python e utiliza o Python-GLPK para a comunicação com o GLPK que é utilizado
para resolver o problema de Programação Linear. O SciPy é utilizado para
resolver os problemas de Programação Não-Linear que aparecem na fase de
processamento.

Para a criação dos testes, utilizou-se
\begin{enumerate}
    \item Inkscape e
    \item ImageMagick
\end{enumerate}
e o formato \texttt{ppm} para armazenar os mapas dos testes.

\subsection{Implementação do problema sem obstáculo}
O arquivo \texttt{ppm} é lido e ``comprimido'' de acordo com o tamanho da malha
desejada. Com as informações do arquivo na memória. Verifica-se quais os pares
cuja distância é inferior a máxima permitida e esses pares são salvos em um
arquivo binário (para o caso de precisar utilizá-los novamente).

As informações do arquivo binário são lidas e a partir delas é montado o
problema de Programação Linear que posteriormente é resolvido pelo GLPK.

\subsection{Implementação do problema com obstáculo circular}
Ocorre de maneira semelhante ao problema sem obstáculo exceto que procura-se
por triplas, $(x, z, y)$, que satisfazem 
\begin{itemize}
    \item $d(x, z) < D$,
    \item $d(y, z) < D$,
    \item $l(x, z) > r$,
    \item $l(y, z) > r$.
\end{itemize}

Para um par $(x, y)$ é escolhido a tripla $(x, z, y)$ tal que $d(x, z) + d(y,
z)$ é minimo. Para que essa escolha seja feita é resolvido um problema de
Programação Não-Linear por meio da função \texttt{fmin-cobyla} do pacote SciPy.
Maiores
informações sobre a função \texttt{fmin-cobyla} podem ser encontradas no
Apêndice \ref{cobyla}, na documentação do Scipy e em \cite{Powell:2007}.

\subsection{Implementação do problema com obstáculo geral}
Ocorre de maneira semelhante ao problema com obstáculo circular exceto que ao
invés de utilizar a função \texttt{fmim-cobyla} do pacote SciPy para resolver
um problema de Programação Não-Linear é feita uma busca por exaustão utilizando
o algoritmo de Bresenham para verificar quando $l(x, z) < r$ ou $l(y, z) < r$.
