% Copyright (c) 2012 Raniere Silva <r.gaia.cs@gmail.com>
% Copyright (c) 2012 Fernando Cezarino <feolce@gmail.com>
% Copyright (c) 2012 Ana Paula Diniz Marques <anapdinizm@gmail.com>
% Copyright (c) 2012 Camile Kunz <camileknz@gmail.com>
% Copyright (c) 2012 Ana Flavia <anaflavia.c.lima@gmail.com>
%
% This file is part of 'MS480 - 2012S2 - Aterro com Obstáculo'.
%
% 'MS480 - 2012S2 - Aterro com Obstáculo' is licensed under the Creative
% Commons Attribution-ShareAlike 3.0 Unported License. To view a copy of
% this license, visit http://creativecommons.org/licenses/by-sa/3.0/.
%
% 'MS480 - 2012S2 - Aterro com Obstáculo' is distributed in the hope
% that it will be useful, but WITHOUT ANY WARRANTY; without even the
% implied warranty of MERCHANTABILITY or FITNESS FOR A PARTICULAR
% PURPOSE.

\section{Implementa\c{c}\~{a}o computacional}
A parte computacional do projeto foi desenvolvida utilizando o GLPK (GNU Linear
Programming Kit)\nocite{GLPK} e a linguagem de programação Python\nocite{Python}.

As figuras utilizadas para armazenar os mapas foram armazenadas
utilizando o formato ppm.

\subsection{Implementação seguindo os modelos}
Ao implementar \eqref{eq:model_without_obs} foi possível resolver em
% TODO Incluir resultados.

Já ao implementar \eqref{eq:model_with_obs_nl} não foi possível resolver o
problema pois não havia memória suficiente para alocar a matriz de restrições.

\subsection{Heurística}
A seguir apresentamos a heurística utilizada para conseguir uma solução para
\eqref{eq:model_with_obs_nl} dado que não foi possível alocar a matriz de
restrições.

Considere a relaxação de \eqref{eq:model_with_obs_nl} dada por
\begin{align}
    \text{max } & \sum_{x} \sum_{y} \xi_{x, y}, \notag \\
    \text{s.a. } & \xi_{x, y} \geq 0, && \forall x, y, \notag \\
    & \xi_{x, y} = 0, && \forall (x, y) \mid d(x, y) > 2D,
    \label{eq:model_rel} \\
    & \sum_{y} \xi_{x, y} \leq \phi(x), && \forall x, \notag \\
    & \sum_{x} \xi_{x, y} \leq \psi(y), && \forall y, \notag
\end{align}
i.e., por \eqref{eq:model_without_obs} onde a distância máxima para o transporte
é $2D$.

Agora considere que $\xi_{x, y}^*$ seja a solução ótima de \eqref{eq:model_rel},
$S = \left\{ (x^*, y^*) \mid \xi_{x^*, y^*}^* > 0 \right\}$, i.e., o conjunto
das duplas em que ocorre o transporte de terra satisfazendo \eqref{eq:model_rel},
e
\begin{align*}
    V = \left\{ (\hat{x}, \hat{z}, \hat{y}) \mid d(\hat{x}, \hat{z}) < D \text{
    e } d(\hat{z}, \hat{y}) < D \text{ e } l(\hat{x}, \hat{z}) > r \text{ e }
    l(\hat{z}, \hat{y}) > r \right\},
\end{align*}
i.e., o conjunto das triplas em que é possível ocorrer o transporte de terra
para \eqref{eq:model_with_obs_nl}.

Então, o conjunto dado por
\begin{align*}
    \left\{ (\tilde{x}, \tilde{z}, \tilde{y}) \mid 
    (x^*, z^*, y^*) = (\hat{x}, \hat{z}, \hat{y}), \forall (x^*, y^*) \in S,
    \forall z^*, \forall (\hat{x}, \hat{z}, \hat{y}) \in V \right\}
\end{align*}
é uma boa solução para \eqref{eq:model_with_obs_nl}.
