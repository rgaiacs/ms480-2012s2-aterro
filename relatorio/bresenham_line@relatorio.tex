% Copyright (c) 2012 Raniere Silva <r.gaia.cs@gmail.com>
% Copyright (c) 2012 Fernando Cezarino <feolce@gmail.com>
% Copyright (c) 2012 Ana Paula Diniz Marques <anapdinizm@gmail.com>
% Copyright (c) 2012 Camile Kunz <camileknz@gmail.com>
% Copyright (c) 2012 Ana Flavia <anaflavia.c.lima@gmail.com>
%
% This file is part of 'MS480 - 2012S2 - Aterro com Obstáculo'.
%
% 'MS480 - 2012S2 - Aterro com Obstáculo' is licensed under the Creative
% Commons Attribution-ShareAlike 3.0 Unported License. To view a copy of
% this license, visit http://creativecommons.org/licenses/by-sa/3.0/.
%
% 'MS480 - 2012S2 - Aterro com Obstáculo' is distributed in the hope
% that it will be useful, but WITHOUT ANY WARRANTY; without even the
% implied warranty of MERCHANTABILITY or FITNESS FOR A PARTICULAR
% PURPOSE.

\section{Algoritmo de Bresenham} \label{sse:bresenham_line}
O algoritmo descrito nessa seção determina quais pontos em uma malha
bidimensional devem ser graficados de modo a obter uma boa aproximação
para uma linha reta ligando dois pontos dados.

A seguinte convenção será usada:
\begin{itemize}
\item a célula mais àcima e mais à esquerda da malha é representada por $(0,0)$,
e as coordenadas crescem para a direita e para baixo,
\item as células são representadas por números inteiros, e
\item os pontos extremos do segmento de reta a ser aproximada são $(x_0,y_0)$
e  $(x_1,y_1)$, onde a primeira coordenada representa a coluna e a sgunda, a
linha.
\end{itemize}

O algoritmo será apresentado apenas para segmentos de reta com coeficiente
angular entre $-1$ e $0$, ou seja, no octante delimitado pelas direções
\emph{Oeste} e \emph{Sudoeste}. Neste octante, para cada coluna $x$ entre $x_0$
e $x_1$, o algoritmo determina exatamente uma linha $y$, que conterá uma célula
da aproximação do segmento de reta, enquanto cada linha entre $y_0$ e $y_1$
pode conter várias células da aproximação.

O \emph{Algoritmo de Bresenham} escolhe um inteiro $y$ correspondendo à célula
que mais se aproxima do valor real da função para cada $x$. Assim, para
sucessivas colunas, o valor de $y$ pode continuar o mesmo ou ser incrementado
em $1$.

A equação geral de uma reta passando pelos pontos $(x_0,y_0)$ e  $(x_1,y_1)$
é dada por:
\begin{equation*}
\frac{y-y_0}{y_1-y_0}=\frac{x-x_0}{x_1-x_0}\,.
\end{equation*}
Para cada coluna $x$, a linha cuja célula deve ser escolhida é dada pela
aproximação para o inteiro mais próximo de:
\begin{equation*}
y=\frac{y_1-y_0}{x_1-x_0}(x-x_0) + y_0\,.
\end{equation*}

O coeficiente angular $\frac{y_1-y_0}{x_1-x_0}$ depende apenas dos pontos
extremos, podendo então ser pré-calculado. Assim, $y$ pode ser calculado para
sucessivos inteiros $x$ começando de $y_0$ e repetidamente adicionando o
coeficiente angular.

O erro entre o $y$ aproximado e o verdadeiro estará sempre entre $-0.5$ e
$0.5$. Cada vez que $x$ é incrementado em $1$, o erro é incrementado pelo valor
do coeficiente angular; se ele excede $0.5$, então a aproximação de $y$ é
acrescida em $1$ e o erro é decrescido em $1$.

%TODO colocar o pseudo-código
%TODO colocar versões otimizadas ???
%TODO colocar um exemplo
\nocite{wiki:Bresenham_line}