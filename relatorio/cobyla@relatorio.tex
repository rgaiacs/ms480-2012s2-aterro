% Copyright (c) 2012 Raniere Silva <r.gaia.cs@gmail.com>
% Copyright (c) 2012 Fernando Cezarino <feolce@gmail.com>
% Copyright (c) 2012 Ana Paula Diniz Marques <anapdinizm@gmail.com>
% Copyright (c) 2012 Ana Flavia <anaflavia.c.lima@gmail.com>
%
% This file is part of 'MS480 - 2012S2 - Aterro com Obstáculo'.
%
% 'MS480 - 2012S2 - Aterro com Obstáculo' is licensed under the Creative
% Commons Attribution-ShareAlike 3.0 Unported License. To view a copy of
% this license, visit http://creativecommons.org/licenses/by-sa/3.0/.
%
% 'MS480 - 2012S2 - Aterro com Obstáculo' is distributed in the hope
% that it will be useful, but WITHOUT ANY WARRANTY; without even the
% implied warranty of MERCHANTABILITY or FITNESS FOR A PARTICULAR
% PURPOSE.

\section{COBYLA} \label{cobyla}
A rotina COBYLA \emph{(Constrained Optimization BY Linear Approximation)}
minimiza uma função $f: \mathbb{R}^n \to \mathbb{R}$ sujeito à $m$ restrições de
desigualdade do tipo $g(x) \leq 0$, onde $g: \mathbb{R}^n \to \mathbb{R}$, sem
utilizar nenhuma derivada.

O algoritmo emprega, ao invés de um modelo quadrático, aproximações lineares,
definidas por uma interpolações lineares de $n+1$ pontos, para a função objetivo
e para as restrições.

% O parâmetro $\rho$ controla o tamanho do simplex, e é reduzido automaticamente
% de $\rho_i$ para $\rho_f$. Para cada $\rho$, a subrotina tenta alcançar um bom
% vetor de variáveis para o tamanho atual, e então $\rho$ é reduzido até o valor
% $\rho_f$ ser alcançado.
% 
% Portanto, para $\rho_i$ e $\rho_f$ devem ser designados valores razoáveis de
% mudança inicial para e precisão desejada nas variáveis, respectivamente. Porém,
% essa precisão deve ser vista como objeto para experimentação, pois ela não é
% garantida.
% 
% A rotina trata cada restrição individualmente quando calcula a mudança para as
% variáveis, ao invés de juntar todas as restrições em uma única função de
% penalidade.
