% Copyright (c) 2012 Raniere Silva <r.gaia.cs@gmail.com>
% Copyright (c) 2012 Fernando Cezarino <feolce@gmail.com>
% Copyright (c) 2012 Ana Paula Diniz Marques <anapdinizm@gmail.com>
% Copyright (c) 2012 Ana Flavia <anaflavia.c.lima@gmail.com>
%
% This file is part of 'MS480 - 2012S2 - Aterro com Obstáculo'.
%
% 'MS480 - 2012S2 - Aterro com Obstáculo' is licensed under the Creative
% Commons Attribution-ShareAlike 3.0 Unported License. To view a copy of
% this license, visit http://creativecommons.org/licenses/by-sa/3.0/.
%
% 'MS480 - 2012S2 - Aterro com Obstáculo' is distributed in the hope
% that it will be useful, but WITHOUT ANY WARRANTY; without even the
% implied warranty of MERCHANTABILITY or FITNESS FOR A PARTICULAR
% PURPOSE.

\section{COBYLA}

A rotina COBYLA \emph{(Constrained Optimization BY Linear Approximation)}
minimiza uma função objetivo $f(x)$ sujeito à $m$ restrições de desigualdade
do tipo $g(x)\leq0$, com $x\in\mathbb{R}^n$.

O algoritmo emprega aproximações lineares para a função objetivo e para as
restrições, tais aproximações sendo formadas por interpolações lineares em
$n+1$ pontos no espaço das variáveis. Os pontos da interpolação são tratados
como os vértices de um simplex\footnote{Politopo com $n+1$ vértices em um
espaço $n-$dimensional}.

O parâmetro $\rho$ controla o tamanho do simplex, e é reduzido automaticamente
de $\rho_i$ para $\rho_f$. Para cada $\rho$, a subrotina tenta alcançar um bom
vetor de variáveis para o tamanho atual, e então $\rho$ é reduzido até o valor
$\rho_f$ ser alcançado.

Portanto, para $\rho_i$ e $\rho_f$ devem ser designados valores razoáveis de
mudança inicial para e precisão desejada nas variáveis, respectivamente. Porém,
essa precisão deve ser vista como objeto para experimentação, pois ela não é
garantida.

A rotina trata cada restrição individualmente quando calcula a mudança para as
variáveis, ao invés de juntar todas as restrições em uma única função de
penalidade.