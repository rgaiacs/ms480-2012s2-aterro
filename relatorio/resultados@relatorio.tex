\section{Resultados computacionais}
As tabelas com os dados referentes aos testes computacionais encontrão-se no
Apêndice~\ref{sse:tab:resul_comp}. A seguir encontra-se a análise dos mesmos.

Em relação aos problemas estudados, observamos pela Tabela~\ref{tab:solver} que
o problema com o maior valor da função objetivo é o problema com obstáculo geral
(devido as várias aproximação realizadas e uso de dois canos de comprimento
máximo $D$) e que o problema com o menor valor da função objetivo foi o problema
sem obstáculo (devido a ser utilizado apenas um cano de comprimento máximo $D$).

Pelo Teorema~1 em \cite{Andjel:1989:TP}, esperava-se que 
\begin{align*}
    F(d^l) \geq F(d^c) \geq F(d^u),
\end{align*}
onde $F(d)$ é o valor da função objetivo ao utilizar a distância do tipo $d$.
Essa relação é verificada na Tabela~\ref{tab:dtype}.

Em relação ao comprimento máximo dos canos, esperava-se que ao aumentar o
comprimento, o valor da função também aumentaria. É possível observar isso na
Tabela~\ref{tab:maxd}. Na mesma tabela observa-se que para $D = 900$ e $D =
1000$, o mesmo valor da função objetivo é obtido. Acredita-se que isso decorre
do valor ótimo para o problema ter sido atingido.

E em relação a malha retangular utilizada, esperava-se que quanto menores forem
os retângulos (menor valor de redução), maior fosse o valor da função objetivo.
Essa relação é observada na Tabela~\ref{tab:reduce}.
