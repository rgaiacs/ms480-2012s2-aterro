% Copyright (c) 2012 Raniere Silva <r.gaia.cs@gmail.com>
% Copyright (c) 2012 Fernando Cezarino <feolce@gmail.com>
% Copyright (c) 2012 Ana Paula Diniz Marques <anapdinizm@gmail.com>
% Copyright (c) 2012 Camile Kunz <camileknz@gmail.com>
% Copyright (c) 2012 Ana Flavia <anaflavia.c.lima@gmail.com>
%
% This file is part of 'MS480 - 2012S2 - Aterro com Obstáculo'.
%
% 'MS480 - 2012S2 - Aterro com Obstáculo' is licensed under the Creative
% Commons Attribution-ShareAlike 3.0 Unported License. To view a copy of
% this license, visit http://creativecommons.org/licenses/by-sa/3.0/.
%
% 'MS480 - 2012S2 - Aterro com Obstáculo' is distributed in the hope
% that it will be useful, but WITHOUT ANY WARRANTY; without even the
% implied warranty of MERCHANTABILITY or FITNESS FOR A PARTICULAR
% PURPOSE.

\section{O Problema}
O problema tratado neste trabalho, denominado de ``Aterro com obstáculo'', é uma
versão modificada do problema abordado por Enrique D. Andjel, Tarcísio L. Lopes
e José Mario Martinez em ``A Linear Continuous Transportation Problem''
\cite{Andjel:1989:TP}.

O problema abordado por Andjel é descrito como:
\begin{quotation}
    Transportar terra de uma região $J$ para uma região $A$, sendo $J$ e $A$
    regiões disjuntas, respeitando o volume máximo de terra que pode ser
    retirado de $J$ e o volume máximo de terra que é admitido em $A$. O
    transporte entre dois pontos é realizado por meio de um único cano
    inflexível, i.e., não existe curva, e quando a distância é maior que $D$ o
    transporte é considerado muito caro e por isso não é realizado (tradução
    nossa).
\end{quotation}

A variante tratada neste trabalho consiste em incluir entre $J$ e $A$ uma 
região $P$, distinta de ambas, pela qual os canos que fazem o transporte da
terra não podem passar, i.e., um obstáculo. Nesta variante, o transporte entre
dois pontos é realizado por meio de dois canos inflexíveis e quando a distância
de algum dos canos é maior que $D$, o transporte é considerado muito caro e por
isso não é realizado.
