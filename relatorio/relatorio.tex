% Copyright (c) 2012 Raniere Silva <r.gaia.cs@gmail.com>
% Copyright (c) 2012 Fernando Cezarino <feolce@gmail.com>
% Copyright (c) 2012 Ana Paula Diniz Marques <anapdinizm@gmail.com>
% Copyright (c) 2012 Ana Flavia <anaflavia.c.lima@gmail.com>
%
% This file is part of 'MS480 - 2012S2 - Aterro com Obstáculo'.
%
% 'MS480 - 2012S2 - Aterro com Obstáculo' is licensed under the Creative Commons
% Attribution-ShareAlike 3.0 Unported License. To view a copy of this license,
% visit http://creativecommons.org/licenses/by-sa/3.0/.
%
% 'MS480 - 2012S2 - Aterro com Obstáculo' is distributed in the hope that it
% will be useful, but WITHOUT ANY WARRANTY; without even the implied warranty of
% MERCHANTABILITY or FITNESS FOR A PARTICULAR PURPOSE.
\documentclass[12pt,a4paper]{article}
%Pacotes utilizados
\usepackage[top=3cm,left=2cm,right=2cm,bottom=3cm]{geometry}
\usepackage[utf8]{inputenc}
\usepackage[T1]{fontenc}
\usepackage[brazil]{babel}
\usepackage{indentfirst}
\usepackage{amsmath, amsfonts, amssymb, amsthm}
\usepackage{graphicx}
\usepackage{tikz}
\usetikzlibrary{patterns}
\usepackage{url}
\usepackage{hyperref}
\usepackage{listings}

\newcommand{\flang}[1]{\textit{#1}}
\newcommand{\clang}[1]{\texttt{#1}}

\begin{document}
\title{MS480 - Aterro com obst\'{a}culo}
\author{Ana Flávia da Cunha Lima\footnote{ra093370, \url{anaflavia.c.lima
@gmail.com}} \and Ana Paula Diniz Marques\footnote{ra076433, \url{anapdinizm
@gmail.com}} \and Fernando de Oliveira Cezarino\footnote{ra085855, 
\url{feolce@gmail.com}} \and  Raniere Gaia Costa da Silva\footnote{ra092767, 
\url{r.gaia.cs@gmail.com}}}
\maketitle
\begin{abstract}
    Neste projeto
    %TODO Escrever o resumo.
\end{abstract}
\section*{Licen\c{c}a}
Este trabalho \'{e} licenciado sob a Licen\c{c}a Creative Commons
Atribui\c{c}\~{a}o 3.0 N\~{a}o Adaptada License. Para ver uma c\'{o}pia desta
licen\c{c}a, visite \url{http://creativecommons.org/licenses/by/3.0/}.
\begin{center}
    \includegraphics{../figuras/cc-by.png}
\end{center}
\newpage
\tableofcontents
\newpage

% Copyright (c) 2012 Raniere Silva <r.gaia.cs@gmail.com>
% Copyright (c) 2012 Fernando Cezarino <feolce@gmail.com>
% Copyright (c) 2012 Ana Paula Diniz Marques <anapdinizm@gmail.com>
% Copyright (c) 2012 Camile Kunz <camileknz@gmail.com>
% Copyright (c) 2012 Ana Flavia <anaflavia.c.lima@gmail.com>
%
% This file is part of 'MS480 - 2012S2 - Aterro com Obstáculo'.
%
% 'MS480 - 2012S2 - Aterro com Obstáculo' is licensed under the Creative
% Commons Attribution-ShareAlike 3.0 Unported License. To view a copy of
% this license, visit http://creativecommons.org/licenses/by-sa/3.0/.
%
% 'MS480 - 2012S2 - Aterro com Obstáculo' is distributed in the hope
% that it will be useful, but WITHOUT ANY WARRANTY; without even the
% implied warranty of MERCHANTABILITY or FITNESS FOR A PARTICULAR
% PURPOSE.

\section{O Problema}
O problema tratado neste trabalho, denominado de ``Aterro com obstáculo'', é uma
versão modificada do problema abordado por Enrique D. Andjel, Tarcísio L. Lopes
e José Mario Martinez em ``A Linear Continuous Transportation Problem''
\cite{Andjel:1989:TP}.

O problema abordado por Andjel é descrito como:
\begin{quotation}
    Transportar terra de uma região $J$ para uma região $A$, sendo $J$ e $A$
    regiões disjuntas, respeitando o volume máximo de terra que pode ser
    retirado de $J$ e o volume máximo de terra que é admitido em $A$. O
    transporte entre dois pontos é realizado por meio de um único cano
    inflexível, i.e., não existe curva, e quando a distância é maior que $D$ o
    transporte é considerado muito caro e por isso não é realizado (tradução
    nossa).
\end{quotation}

A variante tratada neste trabalho consiste em incluir entre $J$ e $A$ uma 
região $P$, distinta de ambas, pela qual os canos que fazem o transporte da
terra não podem passar, i.e., um obstáculo. Nesta variante, o transporte entre
dois pontos é realizado por meio de dois canos inflexíveis e quando a distância
de algum dos canos é maior que $D$, o transporte é considerado muito caro e por
isso não é realizado.

% Copyright (c) 2012 Raniere Silva <r.gaia.cs@gmail.com>
% Copyright (c) 2012 Fernando Cezarino <feolce@gmail.com>
% Copyright (c) 2012 Ana Paula Diniz Marques <anapdinizm@gmail.com>
% Copyright (c) 2012 Camile Kunz <camileknz@gmail.com>
% Copyright (c) 2012 Ana Flavia <anaflavia.c.lima@gmail.com>
%
% This file is part of 'MS480 - 2012S2 - Aterro com Obstáculo'.
%
% 'MS480 - 2012S2 - Aterro com Obstáculo' is licensed under the Creative
% Commons Attribution-ShareAlike 3.0 Unported License. To view a copy of
% this license, visit http://creativecommons.org/licenses/by-sa/3.0/.
%
% 'MS480 - 2012S2 - Aterro com Obstáculo' is distributed in the hope
% that it will be useful, but WITHOUT ANY WARRANTY; without even the
% implied warranty of MERCHANTABILITY or FITNESS FOR A PARTICULAR
% PURPOSE.

\section{Modelagem}
A seguir, apresentamos a modelagem proposta por Andjel e expandimos essa
modelagem para o problema a ser tratado.

\subsection{Problema sem obstáculo}
Considere a Figura~\ref{fig:ilust_J_A} em que é ilustrado as regiões $J$ e
$A$.
\begin{figure}[!htb]
    \centering
    \begin{tikzpicture}
        \draw[pattern=bricks] (0,0) -- (0,3) -- (2,2) -- (4,0) -- (0,0);
        \node[fill=white] at (1,1) {$A$};
        \draw[pattern=checkerboard] (6,6) -- (8,6) -- (8,8) -- (6,8) -- (5,7) --
        (6,6);
        \node[fill=white] at (7,7) {$J$};

        % Escala
        \node[fill=white, above] at (5,1.2) {$0$};
        \draw[fill=black] (5,1) rectangle (6,1.2) node[fill=white, above]{$e$};
        \draw[fill=white] (6,1) rectangle (7,1.2) node[fill=white, above]{$2e$};
        \draw[fill=black] (7,1) rectangle (8,1.2) node[fill=white, above]{$3e$};

        % Eixos
        \draw[->] (-.2,8) -- (8.2,8);
        \draw[->] (0,8.2) -- (0,-.2);
    \end{tikzpicture}
    \caption{Ilustra\c{c}\~{a}o das regiões $J$ e $A$.}
    \label{fig:ilust_J_A}
\end{figure}

Para modelarmos o problema, discretizamos a região ilustrada na
Figura~\ref{fig:ilust_J_A} utilizando uma malha quadriculada, como ilustrado na
Figura~\ref{fig:disc_J_A}, e nomeamos cada quadrado da malha por um número
seguindo a lógica parcialmente indicada.
\begin{figure}[!htb]
    \centering
    \begin{tikzpicture}
        \draw[color=gray, step=.5] (0,0) grid (8,8);
        \draw[pattern=bricks] (0,0) -- (0,3) -- (2,2) -- (4,0) -- (0,0);
        \node[fill=white] at (1,1) {$A$};
        \draw[pattern=checkerboard] (6,6) -- (8,6) -- (8,8) -- (6,8) -- (5,7) --
        (6,6);
        \node[fill=white] at (7,7) {$J$};

        \draw[pattern=bricks](0,0) rectangle (2,2) node[fill=white,
        midway]{$A$};
        \draw[pattern=checkerboard] (6,6) rectangle (8,8) node[fill=white,
        midway]{$J$};

        \foreach \x in {1,...,11}{
            \node at (\x/2 - .25,7.75) {\small{$\x$}};
        }
        \foreach \x in {17,...,26}{
            \node at (\x/2 - 8.25,7.25) {\small{$\x$}};
        }

        % Escala
        \node[fill=white, above] at (5,1.2) {$0$};
        \draw[fill=black] (5,1) rectangle (6,1.2) node[fill=white, above]{$e$};
        \draw[fill=white] (6,1) rectangle (7,1.2) node[fill=white, above]{$2e$};
        \draw[fill=black] (7,1) rectangle (8,1.2) node[fill=white, above]{$3e$};

        % Eixos
        \draw[->] (-.2,8) -- (8.2,8);
        \draw[->] (0,8.2) -- (0,-.2);
    \end{tikzpicture}
    \caption{Ilustra\c{c}\~{a}o das regiões $J$ e $A$.}
    \label{fig:disc_J_A}
\end{figure}

É possível observar que alguns elementos da malha quadriculada encontram-se
apenas parcialmente dentro de uma das regiões. Para corrigir isso,
aproximamos as regiões para a malha utilizada de modo que todos os
elementos da malha que encontram-se parcialmente presentes em uma das regiões
passa a pertencer totalmente a ela. O resultado é apresentado na
Figura~\ref{fig:disc_J_A_aprox}.
%TODO Melhorar esse parágrafo.

\begin{figure}[!htb]
    \centering
    \begin{tikzpicture}
        \draw[color=gray, step=.5] (0,0) grid (8,8);
        \draw[pattern=bricks] (0,0) -- (0,3) -- ++(1,0) -- ++(0,-.5) -- ++(1,0)
        -- ++(0,-.5) -- ++(.5,0) -- ++(0,-.5) -- ++(.5,0) -- ++(0,-.5) --
        ++(.5,0) -- ++(0,-.5) -- ++(.5,0) -- ++(0,-.5) -- (4,0) -- (0,0);
        \node[fill=white] at (1,1) {$A$};
        \draw[pattern=checkerboard] (6,6) -- (8,6) -- (8,8) -- (6,8) --
        ++(-.5,0) -- ++(0,-.5) -- ++(-.5,0) -- ++ (0,-.5) -- ++(0,-.5) --
        ++(.5,0) -- ++(0,-.5) -- ++(.5,0) -- (6,6);
        \node[fill=white] at (7,7) {$J$};

        \draw[pattern=bricks](0,0) rectangle (2,2) node[fill=white,
        midway]{$A$};
        \draw[pattern=checkerboard] (6,6) rectangle (8,8) node[fill=white,
        midway]{$J$};

        % Escala
        \node[fill=white, above] at (5,1.2) {$0$};
        \draw[fill=black] (5,1) rectangle (6,1.2) node[fill=white, above]{$e$};
        \draw[fill=white] (6,1) rectangle (7,1.2) node[fill=white, above]{$2e$};
        \draw[fill=black] (7,1) rectangle (8,1.2) node[fill=white, above]{$3e$};

        % Eixos
        \draw[->] (-.2,8) -- (8.2,8);
        \draw[->] (0,8.2) -- (0,-.2);
    \end{tikzpicture}
    \caption{Ilustra\c{c}\~{a}o das regiões $J$ e $A$.}
    \label{fig:disc_J_A_aprox}
\end{figure}

Seja $\phi : J \to \mathbb{R}$, $\psi: A \to \mathbb{R}$, $f: (J, A) \to
\mathbb{R}$ e $d: (J, A) \to \mathbb{R}$, onde $\phi(x)$ corresponde ao volume
de terra disponível no quadrado $x$, $\psi(y)$ ao volume de terra admitido no
quadrado $y$, $f(x, y)$ ao volume de terra transportado do quadrado $x$ para o
quadrado $y$ e $d(x, y)$ a distância entre os quadrados $x$ e $y$.

Sendo $\xi_{x, y} = f(x, y)$ a variável de decisão, podemos modelar o problema
da seguinte forma:
\begin{subequations}
    \begin{align}
        \text{max } & \sum_{x} \sum_{y} \xi_{x, y},
        \label{eq:model_without_obs:obj_func} \\
        \text{s.a. } & \xi_{x, y} \geq 0, && \forall x, y,
        \label{eq:model_without_obs:var} \\
        & \xi_{x, y} = 0, && \forall (x, y) \mid d(x, y) > D,
        \label{eq:model_without_obs:max_dist} \\
        & \sum_{y} \xi_{x, y} \leq \phi(x), && \forall x,
        \label{eq:model_without_obs:max_jazida} \\
        & \sum_{x} \xi_{x, y} \leq \psi(y), && \forall y,
        \label{eq:model_without_obs:max_aterro}
    \end{align}
    \label{eq:model_without_obs}
\end{subequations}
onde $x \in J$, $y \in A$, \eqref{eq:model_without_obs:max_dist} refere-se a
distância máxima dos canos utilizados, \eqref{eq:model_without_obs:max_jazida} a
quantidade de terra que pode ser retirado de $J$ e
\eqref{eq:model_without_obs:max_aterro} a quantidade de terra necessária em $A$.

O modelo apresentado em \eqref{eq:model_without_obs} possui função objetivo e
restrições lineares. Em relação à restrição
\eqref{eq:model_without_obs:max_dist}, $d(x, y)$ é um dado do problema.

\subsection{Problema com obstáculo circular}
Considere que a região $R$ pela qual os canos que fazem o transporte da terra
não podem passar é circular, com centro $c = (c_1, c_2) \in \mathbb{R}^2$ e raio
$r \in \mathbb{R}$. Então temos a região de interesse ilustrada na
Figura~\ref{fig:disc_J_A_R}.
\begin{figure}[!htb]
    \centering
    \begin{tikzpicture}
        \draw[color=gray, step=.5] (0,0) grid (8,8);

        \draw[color=gray, step=.5] (0,0) grid (8,8);
        \draw[pattern=bricks] (0,0) -- (0,3) -- ++(1,0) -- ++(0,-.5) -- ++(1,0)
        -- ++(0,-.5) -- ++(.5,0) -- ++(0,-.5) -- ++(.5,0) -- ++(0,-.5) --
        ++(.5,0) -- ++(0,-.5) -- ++(.5,0) -- ++(0,-.5) -- (4,0) -- (0,0);
        \node[fill=white] at (1,1) {$A$};
        \draw[pattern=checkerboard] (6,6) -- (8,6) -- (8,8) -- (6,8) --
        ++(-.5,0) -- ++(0,-.5) -- ++(-.5,0) -- ++ (0,-.5) -- ++(0,-.5) --
        ++(.5,0) -- ++(0,-.5) -- ++(.5,0) -- (6,6);
        \node[fill=white] at (7,7) {$J$};
        \draw[pattern=fivepointed stars] (4,4) node[fill=white]{$R$}
        circle (1.7) ;

        % Escala
        \node[fill=white, above] at (5,1.2) {$0$};
        \draw[fill=black] (5,1) rectangle (6,1.2) node[fill=white, above]{$e$};
        \draw[fill=white] (6,1) rectangle (7,1.2) node[fill=white, above]{$2e$};
        \draw[fill=black] (7,1) rectangle (8,1.2) node[fill=white, above]{$3e$};

        % Eixo
        \draw[->] (-.2,8) -- (8.2,8);
        \draw[->] (0,8.2) -- (0,-.2);
    \end{tikzpicture}
    \caption{Ilustra\c{c}\~{a}o da malha quadricular sobre a regi\~{a}o de
    interesse.}
    \label{fig:disc_J_A_R}
\end{figure}
%TODO arrumar fig:disc_J_A_R

Seja $H$ a região que não corresponde \`{a} jazida, nem ao aterro e nem ao
obstáculo. De maneira semelhante \`{a}s definições utilizadas na subseção
anterior, seja $\phi: J \to \mathbb{R}$, $\psi: A \to \mathbb{R}$,
$f: (J, H, A) \to \mathbb{R}$, $d: (J \cup A, H) \to \mathbb{R}$ e $l:
(J \cup A, H) \to \mathbb{R}$, onde $\phi(x)$ corresponde ao volume de terra
disponível no quadrado $x$, $\psi(y)$ ao volume de terra necessário no quadrado
$y$, $f(x, z, y)$ ao volume de terra transportado do quadrado $x$ para o
quadrado $y$ passando pelo quadrado $z$, $d(w, z)$ a distância\footnote{A
distância é definida no Apêndice~\ref{sse:distance}.} entre os
quadrados $w$ e $z$ e $l(w, z)$ a distância entre a reta
formada pelos quadrados $w$ e $z$ ao ponto $c$.
%TODO ainda usamos distância entre ponto e reta?

Sendo $\xi_{x, z, y} = f(x, z, y)$ a variável de decisão, podemos modelar o
problema da seguinte forma:
\begin{subequations}
    \begin{align}
        \text{max } & \sum_{x} \sum_{z} \sum_{y} \xi_{x, z, y},
        \label{eq:model_with_obs_nl:obj_func} \\
        \text{s.a. } & \xi_{x, z, y} \geq 0, && \forall x, z, y,
        \label{eq:model_with_obs_nl:var} \\
        & \xi_{x, z, y} = 0, && \forall (x, z, y) \mid d(x, z) > D \text{ ou } 
        d(y, z) > D,
        \label{eq:model_with_obs_nl:max_dist} \\
        & \xi_{x, z, y} = 0, && \forall (x, z, y) \mid l(x, z) < r \text{ ou }
        l(y, z) < r,
        \label{eq:model_with_obs_nl:dist_obs} \\
        & \sum_{z} \sum_{y} \xi_{x, z, y} \leq \phi(x), && \forall x,
        \label{eq:model_with_obs_nl:max_jazida} \\
        & \sum_{x} \sum_{z} \xi_{x, z, y} \leq \psi(y), && \forall y,
        \label{eq:model_with_obs_nl:max_aterro}
    \end{align}
    \label{eq:model_with_obs_nl}
\end{subequations}
onde $x \in J$, $y \in A$, $z \in H$, \eqref{eq:model_with_obs_nl:max_dist}
refere-se a distância máxima dos canos utilizados,
\eqref{eq:model_with_obs_nl:dist_obs} a restrição dos canos não passar pela
região $R$, \eqref{eq:model_with_obs_nl:max_jazida} a quantidade de terra que pode
ser retirado de $J$ e \eqref{eq:model_with_obs_nl:max_aterro} a quantidade de terra
necessário em $A$.

O modelo apresentado em \eqref{eq:model_with_obs_nl} possue função objetivo e
restrições lineares. Em relação \`{a}s restrições
\eqref{eq:model_with_obs_nl:max_dist} e \eqref{eq:model_with_obs_nl:dist_obs},
$d(x, z)$, $d(y, z)$, $l(x, z)$ e $l(y, z)$ são dados do problema.

\subsection{Problema com obst\'{a}culo geral}
Seria interessante poder lidar com regiões $R$ mais gerais. Uma forma de fazer
isso é aproximar a região $R$ utilizando o mesmo procedimento adotado para
aproximar as regiões $J$ e $A$. O resultado da aproximação da região $R$ é
ilustrada na Figura~\ref{fig:disc_J_A_R_aprox}.
\begin{figure}[!htb]
    \centering
    \begin{tikzpicture}
        \draw[color=gray, step=.5] (0,0) grid (8,8);

        \draw[color=gray, step=.5] (0,0) grid (8,8);
        \draw[pattern=bricks] (0,0) -- (0,3) -- ++(1,0) -- ++(0,-.5) -- ++(1,0)
        -- ++(0,-.5) -- ++(.5,0) -- ++(0,-.5) -- ++(.5,0) -- ++(0,-.5) --
        ++(.5,0) -- ++(0,-.5) -- ++(.5,0) -- ++(0,-.5) -- (4,0) -- (0,0);
        \node[fill=white] at (1,1) {$A$};
        \draw[pattern=checkerboard] (6,6) -- (8,6) -- (8,8) -- (6,8) --
        ++(-.5,0) -- ++(0,-.5) -- ++(-.5,0) -- ++ (0,-.5) -- ++(0,-.5) --
        ++(.5,0) -- ++(0,-.5) -- ++(.5,0) -- (6,6);
        \node[fill=white] at (7,7) {$J$};
        \draw[pattern=fivepointed stars] (3,2.5) -- ++(0,-.5) -- ++(2,0) --
        ++(0,.5) -- ++ (.5,0) -- ++(0,.5) -- ++(.5,0) -- ++(0,2) -- ++ (-.5,0)
        -- ++(0,.5) -- ++(-.5,0) -- ++(0,.5) -- ++(-2,0) -- ++(0,-.5) --
        ++(-.5,0) -- ++(0,-.5) -- ++(-.5,0) -- ++(0,-2) -- ++(.5,0) -- ++(0,-.5)
        -- (3,2.5);
        \node[fill=white] at (4,4) {$R$};

        % Escala
        \node[fill=white, above] at (5,1.2) {$0$};
        \draw[fill=black] (5,1) rectangle (6,1.2) node[fill=white, above]{$e$};
        \draw[fill=white] (6,1) rectangle (7,1.2) node[fill=white, above]{$2e$};
        \draw[fill=black] (7,1) rectangle (8,1.2) node[fill=white, above]{$3e$};

        % Eixo
        \draw[->] (-.2,8) -- (8.2,8);
        \draw[->] (0,8.2) -- (0,-.2);
    \end{tikzpicture}
    \caption{Ilustra\c{c}\~{a}o da malha quadricular sobre a regi\~{a}o de
    interesse.}
    \label{fig:disc_J_A_R_aprox}
\end{figure}
%TODO arrumar fig:disc_J_A_R_aprox

O modelo a ser utilizado com a aproximação da região $R$ é semelhante a
\eqref{eq:model_with_obs_nl} mudando apenas a restrição
\eqref{eq:model_with_obs_nl:dist_obs} que passa a ser
\begin{align}
    % \xi_{x, z, y} &= 0, && \forall (x, z, y) \mid b(x, z) \notin R \text{ ou }
    % b(y, z) \notin R,
    \xi_{x, z, y} &= 0, && \forall (x, z, y) \mid b(x, z) \neq 1 \text{ ou }
    b(y, z) \neq 1,
    \label{eq:model_with_obs:dist_obs} 
\end{align}
onde $b: (J \cup A, H) \to \left\{ 0, 1 \right\}$ é $1$ se a linha reta
construída entre os quadrados $w \in J \cup A$ e $z \in H$ utilizando o
algoritmo de Bresenham\footnote{Ver o Apêndice~\ref{sse:bresenham_line}.} passa
por $R$ e $0$ caso contrário.

Assim com nas restrições \eqref{eq:model_with_obs_nl:max_dist} e
\eqref{eq:model_with_obs_nl:dist_obs}, $d(x, z)$, $d(y, z)$, $l(x, z)$ e $l(y,
z)$ são dados do problema, na restri\c{c}\~{a}o
\eqref{eq:model_with_obs:dist_obs}, $b(x, z)$ e $b(y, z)$ tamb\'{e}m s\~{a}o
dados do problema.

% Copyright (c) 2012 Raniere Silva <r.gaia.cs@gmail.com>
% Copyright (c) 2012 Fernando Cezarino <feolce@gmail.com>
% Copyright (c) 2012 Ana Paula Diniz Marques <anapdinizm@gmail.com>
% Copyright (c) 2012 Camile Kunz <camileknz@gmail.com>
% Copyright (c) 2012 Ana Flavia <anaflavia.c.lima@gmail.com>
%
% This file is part of 'MS480 - 2012S2 - Aterro com Obstáculo'.
%
% 'MS480 - 2012S2 - Aterro com Obstáculo' is licensed under the Creative
% Commons Attribution-ShareAlike 3.0 Unported License. To view a copy of
% this license, visit http://creativecommons.org/licenses/by-sa/3.0/.
%
% 'MS480 - 2012S2 - Aterro com Obstáculo' is distributed in the hope
% that it will be useful, but WITHOUT ANY WARRANTY; without even the
% implied warranty of MERCHANTABILITY or FITNESS FOR A PARTICULAR
% PURPOSE.

\section{Modelagem Aperfei\c{c}oada}
A seguir, aperfei\c{c}oamos as modelagens apresentada na se\c{c}\~{a}o anterior.

\subsection{Problema sem obstáculo}
Na modelagem \eqref{eq:model_without_obs}, utilizamos $|J| |A|$ vari\'{a}veis.
Pela restri\c{c}\~{a}o \eqref{eq:model_without_obs:max_dist} observamos que
parte destas vari\'{a}veis s\~{a}o fixadas em zero uma vez que $d(x, y)$ \'{e}
um dado do problema.

\'{E} poss\'{i}vel construir o problema sem as vari\'{a}veis que s\~{a}o nulas
por $d(x, y) > D$ de modo a reduzir o consumo de mem\'{o}ria e eventualmente
aumentar a velocidade.

\subsection{Problema com obstáculo circular}
Na modelagem \eqref{eq:model_with_obs_nl}, utilizamos $|J| |A| |H|$
vari\'{a}veis. Assim como no problema sem obst\'{a}culo, observamos que pelas
restri\c{c}\~{o}es \eqref{eq:model_with_obs_nl:max_dist} e
\eqref{eq:model_with_obs_nl:dist_obs} parte das vari\'{a}veis s\~{a}o fixadas em
zero.

Novamente, \'{e} poss\'{i}vel construir o modelo sem as vari\'{a}veis que s\~{a}o nulas de
modo a reduzir o consumo de mem\'{o}ria e eventualmente aumentar a velocidade.

Consideremos agora os pontos $\hat{x} \in J$, $\hat{y} \in A$, $\hat{z},
\tilde{z} \in H$, tal que
\begin{align*}
    d(\hat{x}, \hat{z}) &< D, & d(\hat{y}, \hat{z}) &< D, \\
    l(\hat{x}, \hat{z}) &> r, & l(\hat{y}, \hat{z}) &> r, \\
    d(\hat{x}, \tilde{z}) &< D, & d(\hat{y}, \tilde{z}) &< D, \\
    l(\hat{x}, \tilde{z}) &> r, & l(\hat{y}, \tilde{z}) &> r
\end{align*}
e
\begin{align*}
    d(\hat{x}, \hat{z}) + d(\hat{y}, \hat{z}) &< d(\hat{x}, \tilde{z}) +
    d(\hat{y}, \tilde{z}).
\end{align*}
Pela modelagem \eqref{eq:model_with_obs_nl}, verifica-se que
$\xi_{\hat{x},\hat{z},\hat{y}}$ e $\xi_{\hat{x},\tilde{z},\hat{y}}$
possuem o mesmo peso na fun\c{c}\~{a}o objetivo e portanto os seguintes valores
s\~{a}o ``equiprov\'{a}veis'' na solu\c{c}\~{a}o \'{o}tima:
\begin{align*}
    \xi_{\hat{x},\hat{z},\hat{y}} &= \min(\phi(\hat{x}), \psi(\hat{y})), &
    \xi_{\hat{x},\tilde{z},\hat{y}} &= 0, \\
    \xi_{\hat{x},\hat{z},\hat{y}} &= 0, &
    \xi_{\hat{x},\tilde{z},\hat{y}} &= \min(\phi(\hat{x}), \psi(\hat{y})), \\
    \xi_{\hat{x},\hat{z},\hat{y}} &= 0.5 \min(\phi(\hat{x}), \psi(\hat{y})), &
    \xi_{\hat{x},\tilde{z},\hat{y}} &= 0.5 \min(\phi(\hat{x}), \psi(\hat{y})).
\end{align*}

Seria interessante que na solu\c{c}\~{a}o \'{o}tima apenas ocorresse
\begin{align*}
    \xi_{\hat{x},\hat{z},\hat{y}} &= \min(\phi(\hat{x}), \psi(\hat{y})), &
    \xi_{\hat{x},\tilde{z},\hat{y}} &= 0,
\end{align*}
pois neste caso a dist\^{a}ncia entre $\hat{x}$ e $\hat{y}$ \'{e}
``m\'{i}nima''. Uma vez que $d(x, z)$ e $d(y, z)$ s\~{a}o dados do problema, \'{e}
poss\'{i}vel construir o modelo com apenas uma vari\'{a}vel para o par
$(x, y)$ e utilizando $z$ tal que $d(x, z) + d(y, z)$ seja m\'{i}nimo para $(x,
y)$. Ao fazer isso, o n\'{u}mero de vari\'{a}veis do problema com obst\'{a}culo
passa a ser da ordem de $|J| |A|$ que \'{e} a mesma ordem do problema sem
obst\'{a}culo.

\subsection{Problema com obst\'{a}culo geral}
Tudo o que foi dito para o problema com obst\'{a}culo circular \'{e} facilmente
aplicado para o problema com obst\'{a}culo geral usando o algoritmo de
Bresenham.

% Copyright (c) 2012 Raniere Silva <r.gaia.cs@gmail.com>
% Copyright (c) 2012 Fernando Cezarino <feolce@gmail.com>
% Copyright (c) 2012 Ana Paula Diniz Marques <anapdinizm@gmail.com>
% Copyright (c) 2012 Camile Kunz <camileknz@gmail.com>
% Copyright (c) 2012 Ana Flavia <anaflavia.c.lima@gmail.com>
%
% This file is part of 'MS480 - 2012S2 - Aterro com Obstáculo'.
%
% 'MS480 - 2012S2 - Aterro com Obstáculo' is licensed under the Creative
% Commons Attribution-ShareAlike 3.0 Unported License. To view a copy of
% this license, visit http://creativecommons.org/licenses/by-sa/3.0/.
%
% 'MS480 - 2012S2 - Aterro com Obstáculo' is distributed in the hope
% that it will be useful, but WITHOUT ANY WARRANTY; without even the
% implied warranty of MERCHANTABILITY or FITNESS FOR A PARTICULAR
% PURPOSE.

\section{Implementa\c{c}\~{a}o computacional}
A parte computacional do projeto foi desenvolvida utilizando a linguagem de
programação Python\nocite{Python}, o pacote Scipy, o GLPK (GNU Linear
Programming Kit)\nocite{GLPK} e o Python-GLPK.
%TODO usarm emph?

As figuras utilizadas para armazenar os mapas foram armazenadas
utilizando o formato \texttt{ppm}.

\subsection{Implementação do problema sem obstáculo}
Ao implementar \eqref{eq:model_without_obs} foi possível resolver em
% TODO Incluir resultados.

O arquivo \texttt{ppm} é lido e ''comprimido'' de acordo com o tamanho da malha
desejada. Depois, construímos o modelo e por último, resolvemos o problema.
%TODO melhorar esse parágrafo

\subsection{Implementação do problema com obstáculo circular}
Para o subproblema de determinar o ponto $z$ que minimiza $\overline{xyz}$,
utilizou-se a função \texttt{fmin-cobyla} do pacote Scipy, que minimiza uma
função de várias variáveis sujeita a restrições utilizando um algoritmo baseado
em aproximações lineares para a função objetivo e as restrições. Maiores
informações sobre a função \texttt{fmin-cobyla} podem ser encontradas no
Apêndice , na documentação do Scipy e em.
%TODO colocar referências

Depois, disso, prosseguimos como no problema dem obstáculo.

\subsection{Implementação do problema com obstáculo geral}
Utilizou-se o algoritmo de Bresenham.

%TODO melhorar essa seção
% TODO Resultados

\appendix
% Copyright (c) 2012 Raniere Silva <r.gaia.cs@gmail.com>
% Copyright (c) 2012 Fernando Cezarino <feolce@gmail.com>
% Copyright (c) 2012 Ana Paula Diniz Marques <anapdinizm@gmail.com>
% Copyright (c) 2012 Camile Kunz <camileknz@gmail.com>
% Copyright (c) 2012 Ana Flavia <anaflavia.c.lima@gmail.com>
%
% This file is part of 'MS480 - 2012S2 - Aterro com Obstáculo'.
%
% 'MS480 - 2012S2 - Aterro com Obstáculo' is licensed under the Creative
% Commons Attribution-ShareAlike 3.0 Unported License. To view a copy of
% this license, visit http://creativecommons.org/licenses/by-sa/3.0/.
%
% 'MS480 - 2012S2 - Aterro com Obstáculo' is distributed in the hope
% that it will be useful, but WITHOUT ANY WARRANTY; without even the
% implied warranty of MERCHANTABILITY or FITNESS FOR A PARTICULAR
% PURPOSE.

\section{Distância entre pontos} \label{sse:distance}
Ao utilizar a malha retangular, podemos definir a distância entre dois
de seus elementos de pelo menos três maneiras diferentes:
\begin{enumerate}
    \item $d^l$, que é a menor distância entre dois elementos da malha,
    \item $d^u$, que é a maior distância entre dois elementos da malha, e
    \item $d^c$, que é a distância entre os centros de dois elementos da malha.
\end{enumerate}
Na Figura~\ref{fig:dist_malha} é ilustrado cada uma das distâncias acima
descrita.
\begin{figure}[!htb]
    \centering
    \begin{tikzpicture}
        \draw (0,0) rectangle (1,1);
        \draw (2,0) rectangle (3,1);
        \draw[<->] (1,1) -- (2,1) node[midway, below]{$d^l$};

        \draw (4,0) rectangle (5,1);
        \draw (6,0) rectangle (7,1);
        \draw[<->] (4,0) -- (7,1) node[midway, below]{$d^u$};

        \draw (8,0) rectangle (9,1) node[midway](A){};
        \draw (10,0) rectangle (11,1) node[midway](B){};
        \draw[<->] (A.center) -- (B.center) node[midway, below]{$d^c$};
    \end{tikzpicture}
    \caption{Ilustra\c{c}\~{a}o das dist\^{a}ncias entre elementos da malha
    retângular.}
    \label{fig:dist_malha}
\end{figure}

%% Copyright (c) 2012 Raniere Silva <r.gaia.cs@gmail.com>
% Copyright (c) 2012 Fernando Cezarino <feolce@gmail.com>
% Copyright (c) 2012 Ana Paula Diniz Marques <anapdinizm@gmail.com>
% Copyright (c) 2012 Camile Kunz <camileknz@gmail.com>
% Copyright (c) 2012 Ana Flavia <anaflavia.c.lima@gmail.com>
%
% This file is part of 'MS480 - 2012S2 - Aterro com Obstáculo'.
%
% 'MS480 - 2012S2 - Aterro com Obstáculo' is licensed under the Creative
% Commons Attribution-ShareAlike 3.0 Unported License. To view a copy of
% this license, visit http://creativecommons.org/licenses/by-sa/3.0/.
%
% 'MS480 - 2012S2 - Aterro com Obstáculo' is distributed in the hope
% that it will be useful, but WITHOUT ANY WARRANTY; without even the
% implied warranty of MERCHANTABILITY or FITNESS FOR A PARTICULAR
% PURPOSE.

\section{Distância de ponto a reta} \label{sse:point2line}
% TODO Descrever o Algoritmo.

% Copyright (c) 2012 Raniere Silva <r.gaia.cs@gmail.com>
% Copyright (c) 2012 Fernando Cezarino <feolce@gmail.com>
% Copyright (c) 2012 Ana Paula Diniz Marques <anapdinizm@gmail.com>
% Copyright (c) 2012 Camile Kunz <camileknz@gmail.com>
% Copyright (c) 2012 Ana Flavia <anaflavia.c.lima@gmail.com>
%
% This file is part of 'MS480 - 2012S2 - Aterro com Obstáculo'.
%
% 'MS480 - 2012S2 - Aterro com Obstáculo' is licensed under the Creative
% Commons Attribution-ShareAlike 3.0 Unported License. To view a copy of
% this license, visit http://creativecommons.org/licenses/by-sa/3.0/.
%
% 'MS480 - 2012S2 - Aterro com Obstáculo' is distributed in the hope
% that it will be useful, but WITHOUT ANY WARRANTY; without even the
% implied warranty of MERCHANTABILITY or FITNESS FOR A PARTICULAR
% PURPOSE.

\section{Interseção de Círculo com Reta} \label{sse:circle_line}
\nocite{Wolfram:CircleLine}
Considere um círculo, $C$, com centro em $(0, 0)$ e raio $r$ e os pontos $P =
(p_1, p_2)$ e $Q = (q_1, q_2)$. A reta $\overline{PQ}$ pode
\begin{itemize}
    \item não interseptar o círculo $C$,
    \item ser tagente ao círculo $C$, ou
    \item ter dois pontos em comum com o círculo $C$,
\end{itemize}
como ilustrado na Figura~\ref{fig:circle_line}.
\begin{figure}[!htb]
    \centering
    \begin{tikzpicture}[scale=0.8]
        \draw (0,0) circle (2);
        \draw[<->] (0,0) node[below left]{$(0, 0)$} -- +(2, 0)
        node[midway, below]{$r$};
        \draw (-3,3) node[left]{$P$} -- (3,3) node[right]{$Q$};
        \draw (7,0) circle (2);
        \draw[<->] (7,0) node[below left]{$(0, 0)$} -- +(2, 0)
        node[midway, below]{$r$};
        \draw (4,2) node[left]{$P$} -- (10,2) node[right]{$Q$};
        \draw (14,0) circle (2);
        \draw[<->] (14,0) node[below left]{$(0, 0)$} -- +(2, 0)
        node[midway, below]{$r$};
        \draw (11,1) node[left]{$P$} -- (17,1) node[right]{$Q$};
    \end{tikzpicture}
    \caption{Possíveis interseções de um círculo e uma reta. A interseção é
    vazia na figura a esquerda, corresponde a um único ponto na
    figura no centro e  a dois pontos distintos na
    figura a direita.}
    \label{fig:circle_line}
\end{figure}

A equação do círculo é
\begin{align}
    x^2 + y^2 &= r^2
    \label{ec_circ}
\end{align}
e a equação da reta que passa pelos pontos $P$ e $Q$ é
\begin{align*}
    (y-p_2)(q_1-p_1)=(q_2-p_2)(x-p_1)\,.
\end{align*}
Se $p_1\neq q_1$, esta equação pode ser escrita como
\begin{align*}
    y=\left(\frac{q_2-p_2}{q_1-p_1}\right)\left(x-p_1\right)+p_2\,,
\end{align*}
ou ainda
\begin{align*}
    y = \left( \frac{q_2 - p_2}{q_1 - p_1} \right)x + \left(
    \frac{q_1 p_2 - q_2 p_1}{q_1 - p_1} \right).
\end{align*}

Substituíndo em (\ref{ec_circ}), obtemos
\begin{align*}
    x^2  + \left[ \left( \frac{q_2 - p_2}{q_1 - p_1} \right) x + \left(
    \frac{q_1 p_2 - q_2 p_1}{q_1 - p_1} \right) \right]^2  = r^2\,,
\end{align*}
ou ainda
\begin{align*}
    \left[ 1 + \left( \frac{q_2 - p_2}{q_1 - p_1} \right)^2 \right] x^2 +
    2 \left( \frac{q_2 - p_2}{q_1 - p_1} \right) \left( \frac{q_1 p_2 - q_2
    p_1}{q_1 - p_1} \right) x + \left( 
    \frac{q_1 p_2 - q_2 p_1}{q_1 - p_1} \right)^2 & = r^2\,.
\end{align*}
Para a equação do segundo grau anterior, o discriminate é
\begin{align*}
    \Delta &= r^2 \left( \left( q_1 - p_1 \right)^2 + \left( q_2 - p_2
    \right)^2 \right) - \left( p_1 q_2 - p_2 q_1 \right)^2\,;
\end{align*}
e
\begin{itemize}
    \item se $\Delta < 0$ a equação não possui solução real,
    \item se $\Delta = 0$ a equação possui duas soluções reais iguais,
    \item se $\Delta > 0$ a equação possui duas soluções reais distintas.
\end{itemize}
Logo,
\begin{itemize}
    \item $\overline{PQ}$ não interseptar o círculo $C$ quando $\Delta < 0$\,,
    \item $\overline{PQ}$ é tagente ao círculo $C$ quando $\Delta = 0$\,,
    \item $\overline{PQ}$ tem dois pontos em comum com o círculo $C$ quando
        $\Delta > 0$\,.
\end{itemize}

% Copyright (c) 2012 Raniere Silva <r.gaia.cs@gmail.com>
% Copyright (c) 2012 Fernando Cezarino <feolce@gmail.com>
% Copyright (c) 2012 Ana Paula Diniz Marques <anapdinizm@gmail.com>
% Copyright (c) 2012 Camile Kunz <camileknz@gmail.com>
% Copyright (c) 2012 Ana Flavia <anaflavia.c.lima@gmail.com>
%
% This file is part of 'MS480 - 2012S2 - Aterro com Obstáculo'.
%
% 'MS480 - 2012S2 - Aterro com Obstáculo' is licensed under the Creative
% Commons Attribution-ShareAlike 3.0 Unported License. To view a copy of
% this license, visit http://creativecommons.org/licenses/by-sa/3.0/.
%
% 'MS480 - 2012S2 - Aterro com Obstáculo' is distributed in the hope
% that it will be useful, but WITHOUT ANY WARRANTY; without even the
% implied warranty of MERCHANTABILITY or FITNESS FOR A PARTICULAR
% PURPOSE.

\section{Algoritmo de Bresenham} \label{sse:bresenham_line}
O algoritmo descrito nessa seção determina quais pontos em uma malha
bidimensional devem ser graficados de modo a obter uma boa aproximação
para uma linha reta ligando dois pontos dados.

A seguinte convenção será usada:
\begin{itemize}
\item a célula mais àcima e mais à esquerda da malha é representada por $(0,0)$,
e as coordenadas crescem para a direita e para baixo,
\item as células são representadas por números inteiros, e
\item os pontos extremos do segmento de reta a ser aproximada são $(x_0,y_0)$
e  $(x_1,y_1)$, onde a primeira coordenada representa a coluna e a sgunda, a
linha.
\end{itemize}

O algoritmo será apresentado apenas para segmentos de reta com coeficiente
angular entre $-1$ e $0$, ou seja, no octante delimitado pelas direções
\emph{Oeste} e \emph{Sudoeste}. Neste octante, para cada coluna $x$ entre $x_0$
e $x_1$, o algoritmo determina exatamente uma linha $y$, que conterá uma célula
da aproximação do segmento de reta, enquanto cada linha entre $y_0$ e $y_1$
pode conter várias células da aproximação.

O \emph{Algoritmo de Bresenham} escolhe um inteiro $y$ correspondendo à célula
que mais se aproxima do valor real da função para cada $x$. Assim, para
sucessivas colunas, o valor de $y$ pode continuar o mesmo ou ser incrementado
em $1$.

A equação geral de uma reta passando pelos pontos $(x_0,y_0)$ e  $(x_1,y_1)$
é dada por:
\begin{equation*}
\frac{y-y_0}{y_1-y_0}=\frac{x-x_0}{x_1-x_0}\,.
\end{equation*}
Para cada coluna $x$, a linha cuja célula deve ser escolhida é dada pela
aproximação para o inteiro mais próximo de:
\begin{equation*}
y=\frac{y_1-y_0}{x_1-x_0}(x-x_0) + y_0\,.
\end{equation*}

O coeficiente angular $\frac{y_1-y_0}{x_1-x_0}$ depende apenas dos pontos
extremos, podendo então ser pré-calculado. Assim, $y$ pode ser calculado para
sucessivos inteiros $x$ começando de $y_0$ e repetidamente adicionando o
coeficiente angular.

O erro entre o $y$ aproximado e o verdadeiro estará sempre entre $-0.5$ e
$0.5$. Cada vez que $x$ é incrementado em $1$, o erro é incrementado pelo valor
do coeficiente angular; se ele excede $0.5$, então a aproximação de $y$ é
acrescida em $1$ e o erro é decrescido em $1$.

%TODO colocar o pseudo-código
%TODO colocar versões otimizadas ???
%TODO colocar um exemplo
\nocite{wiki:Bresenham_line}
% Copyright (c) 2012 Raniere Silva <r.gaia.cs@gmail.com>
% Copyright (c) 2012 Fernando Cezarino <feolce@gmail.com>
% Copyright (c) 2012 Ana Paula Diniz Marques <anapdinizm@gmail.com>
% Copyright (c) 2012 Camile Kunz <camileknz@gmail.com>
% Copyright (c) 2012 Ana Flavia <anaflavia.c.lima@gmail.com>
%
% This file is part of 'MS480 - 2012S2 - Aterro com Obstáculo'.
%
% 'MS480 - 2012S2 - Aterro com Obstáculo' is licensed under the Creative
% Commons Attribution-ShareAlike 3.0 Unported License. To view a copy of
% this license, visit http://creativecommons.org/licenses/by-sa/3.0/.
%
% 'MS480 - 2012S2 - Aterro com Obstáculo' is distributed in the hope
% that it will be useful, but WITHOUT ANY WARRANTY; without even the
% implied warranty of MERCHANTABILITY or FITNESS FOR A PARTICULAR
% PURPOSE.

\section{Formato Netpbm} \label{sse:Netpbm}
Nesta seção descrevemos o formato de arquivo \clang{ppm} utilizado neste projeto
que é um mapa de pixels capaz de representar todas as cores definidas pelo
padrão \clang{RGB} e um dos formatos descritos na biblioteca
Netpbm.\nocite{wiki:Netpbm_format, Henderson:Netpbm}

Cada arquivo inicia com dois bits (em ASCII) que especifica o formato do arquivo
e a codificação (ASCII ou binário), no caso do arquivo \clang{ppm} utilizados
será \clang{P3}.\footnote{Linhas iniciadas com cerquilha, \#, são tratadas
como comentários.}

Após os dois bits iniciais, é especificado a largura e a altura da imagem e o
valor máximo das cores, que deve ser maior que zero e menor que 65536. Por
último, é apresentado um conjunto de triplas que indicam os valores das cores
vermelho, verde e azul do pixel corespondente, sendo que a ordem de relação
entre as triplas e os pixels é da esquerda para a direita e de cima para baixo.

A seguir é apresentado o conteudo de um arquivo \clang{ppm}. A visualização do
mesmo encontra-se na Figura~\ref{fig:minimal.ppm}.
\lstinputlisting[basicstyle=\ttfamily,]{../src/test/minimal.ppm}
\begin{figure}[!htb]
    \begin{center}
        \includegraphics[scale=20]{../src/test/minimal.png}
    \end{center}
    \caption{Visualização do arquivo \clang{ppm} apresentado na
    Seção~\ref{sse:Netpbm} com um aumento de 20 vezes.}
    \label{fig:minimal.ppm}
\end{figure}



\bibliographystyle{alpha}
\bibliography{../referencias}
\end{document}
