\section{Tabelas com os Resultados computacionais}

\begin{landscape}
% \csvautotabular{../src/test/solver.csv}
\begin{table}
    \centering
    \caption{Comparação dos Métodos de Resolução (utilizando com máxima distância de 800,
    redução de 50 e $d^c$)}
    \label{tab:solver}
    \begin{tabular}{|c|c|c|c|c|c|c|c|c|c|c|}
        \hline
        & \multicolumn{5}{|c|}{t1.svg} & \multicolumn{5}{|c|}{t2.svg} \\ \hline
        Solver & Linhas & Colunas & F. Obj. & T. Proc. & T. Sol. & Linhas &
        Colunas & F. Obj. & T. Proc. & T. Sol. \\ \hline
        aterro & 110 & 817 & 38.0 & 3.5974 & 0.0436 & 140 & 1629 & 55.0 & 3.6807 & 0.0961 \\ \hline
        raterro & 110 & 2989 & 49.0 & 3.5674 & 0.1562 & 140 & 4896 & 68.0 & 3.5937 & 0.2803 \\ \hline
        aaterro & 110 & 1111 & 45.0 & 3.5443 & 0.0635 & 140 & 2128 & 63.0 & 3.4525 & 0.1253 \\ \hline
    \end{tabular}
\end{table}

% \csvautotabular{../src/test/dtype.csv}
\begin{table}
    \centering
    \caption{Comparação dos Tipos de Distâncias (utilizando COBYLA, máxima
    distância de 800 e redução de 50)}
    \label{tab:dtype}
    \begin{tabular}{|c|c|c|c|c|c|c|c|c|c|c|}
        \hline
        & \multicolumn{5}{|c|}{t1.svg} & \multicolumn{5}{|c|}{t2.svg} \\ \hline
        Tipo Dist. & Linhas & Colunas & F. Obj. & T. Proc. & T. Sol. & Linhas &
        Colunas & F. Obj. & T. Proc. & T. Sol. \\ \hline
        0 & 110 & 1578 & 49.0 & 3.4 & 0.0870 & 140 & 2896 & 68.0 & 3.6735 & 0.1699 \\ \hline
        1 & 110 & 2034 & 49.0 & 3.5 & 0.1101 & 140 & 3610 & 68.0 & 3.6662 & 0.2113 \\ \hline
        2 & 110 & 1111 & 45.0 & 3.5 & 0.0635 & 140 & 2128 & 63.0 & 3.4525 & 0.1253 \\ \hline
    \end{tabular}
\end{table}

% \csvautotabular{../src/test/maxd.csv}
\begin{table}
    \centering
    \caption{Comparação do Comprimento Máximo dos Canos (utilizando COBYLA,
    $d^c$ e com redução de 50)}
    \label{tab:maxd}
    \begin{tabular}{|c|c|c|c|c|c|c|c|c|c|c|}
        \hline
        & \multicolumn{5}{|c|}{t1.svg} & \multicolumn{5}{|c|}{t2.svg} \\ \hline
        Max. Dist. & Linhas & Colunas & F. Obj. & T. Proc. & T. Sol. & Linhas &
        Colunas & F. Obj. & T. Proc. & T. Sol. \\ \hline
        800.0 & 110 & 1111 & 45.0 & 3.5443 & 0.0635 & 140 & 2128 & 63.0 & 3.4525 & 0.1253 \\ \hline
        900.0 & 110 & 1776 & 49.0 & 3.4865 & 0.0959 & 140 & 3247 & 68.0 & 3.5934 & 0.1911 \\ \hline
        1000.0 & 110 & 2404 & 49.0 & 3.6357 & 0.1289 & 140 & 4181 & 68.0 & 3.5713 & 0.2497 \\ \hline
    \end{tabular}
\end{table}

% \csvautotabular{../src/test/reduce.csv}
\begin{table}
    \centering
    \caption{Comparação Redução Devido a Malha de Discretização (utilizando COBYLA, máxima
    distância de 800 e $d^c$)}
    \label{tab:reduce}
    \begin{tabular}{|c|c|c|c|c|c|c|c|c|c|c|}
        \hline
        & \multicolumn{5}{|c|}{t1.svg} & \multicolumn{5}{|c|}{t2.svg} \\ \hline
        Redução & Linhas & Colunas & F. Obj. & T. Proc. & T. Sol. & Linhas &
        Colunas & F. Obj. & T. Proc. & T. Sol. \\ \hline
        reduce & rows & cols & value & process & solver & rows & cols & value &
        process & solver \\ \hline
        100 & 34 & 101 & 14.0 & 3.6675 & 0.0059 & 44 & 184 & 20.0 & 3.6143 & 0.0099 \\ \hline
        50 & 110 & 1111 & 45.0 & 3.5443 & 0.0635 & 140 & 2128 & 63.0 & 3.4525 & 0.1253 \\ \hline
    \end{tabular}
\end{table}
\end{landscape}
